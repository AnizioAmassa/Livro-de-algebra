% Pacotes úteis
\usepackage[utf8]{inputenc}
\usepackage[T1]{fontenc}
\usepackage{amsmath, amssymb, amsthm}
\usepackage[most]{tcolorbox}
\usepackage{physics}
\usepackage{tikz}
\usepackage{tikz-3dplot}
\usepackage{tkz-euclide}
\usetikzlibrary{matrix,arrows.meta,decorations.markings}
\usetikzlibrary{calc,angles,quotes}
\usepackage{pgfplots}
\usepackage{amsmath}
\usepackage{enumitem}       % personalizar listas
\usepackage{graphicx}       % imagens
\usepackage{caption}		% legendas de imagens
\usepackage{hyperref}       % links no sumário
\usepackage{geometry}       % margens
\usepackage{mathtools}
\usepackage{listings}
\usepackage{mathrsfs}
\lstset{numbers=left, captionpos=b}
\usepackage{xcolor}
\usepackage{array}
\usepackage{xurl}
\usepackage{fancyhdr} 		% cabeçalhos/rodapés
\usepackage{subfiles}
\usepackage{esvect}
\usepackage{subcaption} % para figuras lado a lado
\usepackage{multicol}
\usepackage[brazil]{babel}
\usepackage{authblk}

\setlength{\parindent}{0pt}
\setlength{\parskip}{0.8em}

\hypersetup{
	colorlinks=true,
	linkcolor=blue,
	urlcolor=blue
}

%Configurações do arquivo
\geometry{margin=2.5cm} % margem
\linespread{1.2}   		% aumenta o espaçamento entre linhas
\usepackage{lmodern} 	% usa fonte Latin Modern


% Criando ambientes (teorema, definicao, exemplo, exercicio e solucao)
\newtheoremstyle{meuTeorema}
{1em}    		% Espaço acima
{1em}    		% Espaço abaixo
{\itshape} 		% Fonte
{}       		% Recuo
{\bfseries} 	% Fonte do cabeçalho
{.}      		% Pontuação após cabeçalho
{.5em}   		% Espaço após cabeçalho
{}       		% Cabeçalho especifico

\theoremstyle{meuTeorema}
\newtheorem{teorema}{Teorema}[section]
\newtheorem{definicao}[teorema]{Definição}
\newtheorem{corolario}[teorema]{Corolário}
\newtheorem{lema}[teorema]{Lema}
\newtheorem{proposicao}[teorema]{Proposição}

\theoremstyle{remark}
\newtheorem{exemplo}[teorema]{Exemplo}


\newcommand{\azul}[1]{\textbf{\textcolor{blue}{#1}}}
\newcommand{\parallelsum}{\mathbin{\!/\mkern-5mu/\!}}
\newcommand{\cblue}[1]{{\color{blue}#1}}
\newcommand{\cred}[1]{{\color{red}#1}}

\newtcolorbox{solucao}[1][]{%
	enhanced,
	breakable,
	colback=brown!5!white,   % creme claro
	colframe=brown!90!black, % borda suave
	fonttitle=\bfseries,
	title=Solução,
	boxrule=0.4pt,
	arc=0pt,
	outer arc=0pt,
	coltitle=black}
