<<<<<<< HEAD
% Pacotes úteis
\usepackage[utf8]{inputenc}
\usepackage[T1]{fontenc}
\usepackage{amsmath, amssymb, amsthm}
\usepackage{physics}
\usepackage{enumitem}       % listas
\usepackage{graphicx}       % imagens
\usepackage{caption}		% legendas nas imagenss
\usepackage{hyperref}       % links no sumário
\usepackage{geometry}       % margens
\usepackage{mathtools}
\usepackage[most]{tcolorbox}% criar ambientes em caixas
\usepackage{xcolor}
\usepackage{listings}

%Configurações do arquivo
\geometry{margin=2.5cm} % margem
\linespread{1.2}   		% aumenta o espaçamento entre linhas
\usepackage{lmodern} 	% usa fonte Latin Modern

% Renomeando comandos
\renewcommand{\lstlistingname}{Código}
\renewcommand{\contentsname}{Sumário}
\renewcommand{\chaptername}{Capítulo}
\renewcommand{\appendixname}{Apêndice}

% Criando ambientes (teorema, definicao, exemplo, exercicio e solucao)
\newtheoremstyle{meuTeorema}
{1em}    		% Espaço acima
{1em}    		% Espaço abaixo
{\itshape} 		% Fonte
{}       		% Recuo
{\bfseries} 	% Fonte do cabeçalho
{.}      		% Pontuação após cabeçalho
{.5em}   		% Espaço após cabeçalho
{}       		% Cabeçalho especifico

\theoremstyle{meuTeorema}
\newtheorem{teorema}{Teorema}[section]

\theoremstyle{definition}
\newtheorem{definicao}{Definição}[section]

\theoremstyle{remark}
\newtheorem{exemplo}{Exemplo}[section]

\newenvironment{exercicio}[1]{%
	\vspace{0.5cm}
	\noindent\textbf{Exercício #1. }\itshape}{\vspace{0.3cm}}

\newtcolorbox{solucao}[1][]{%
	enhanced,
	breakable,
	colback=brown!5!white,   % creme claro
	colframe=brown!90!black, % borda suave
	fonttitle=\bfseries,
	title=Solução,
	boxrule=0.4pt,
	arc=0pt,
	outer arc=0pt,
	coltitle=black
=======
% Pacotes úteis
\usepackage[utf8]{inputenc}
\usepackage[T1]{fontenc}
\usepackage{amsmath, amssymb, amsthm}
\usepackage{physics}
\usepackage{enumitem}       % listas
\usepackage{graphicx}       % imagens
\usepackage{caption}		% legendas nas imagenss
\usepackage{hyperref}       % links no sumário
\usepackage{geometry}       % margens
\usepackage{mathtools}
\usepackage[most]{tcolorbox}% criar ambientes em caixas
\usepackage{xcolor}
\usepackage{listings}

%Configurações do arquivo
\geometry{margin=2.5cm} % margem
\linespread{1.2}   		% aumenta o espaçamento entre linhas
\usepackage{lmodern} 	% usa fonte Latin Modern

% Renomeando comandos
\renewcommand{\lstlistingname}{Código}
\renewcommand{\contentsname}{Sumário}
\renewcommand{\chaptername}{Capítulo}
\renewcommand{\appendixname}{Apêndice}

% Criando ambientes (teorema, definicao, exemplo, exercicio e solucao)
\newtheoremstyle{meuTeorema}
{1em}    		% Espaço acima
{1em}    		% Espaço abaixo
{\itshape} 		% Fonte
{}       		% Recuo
{\bfseries} 	% Fonte do cabeçalho
{.}      		% Pontuação após cabeçalho
{.5em}   		% Espaço após cabeçalho
{}       		% Cabeçalho especifico

\theoremstyle{meuTeorema}
\newtheorem{teorema}{Teorema}[section]

\theoremstyle{definition}
\newtheorem{definicao}{Definição}[section]

\theoremstyle{remark}
\newtheorem{exemplo}{Exemplo}[section]

\newenvironment{exercicio}[1]{%
	\vspace{0.5cm}
	\noindent\textbf{Exercício #1. }\itshape}{\vspace{0.3cm}}

\newtcolorbox{solucao}[1][]{%
	enhanced,
	breakable,
	colback=brown!5!white,   % creme claro
	colframe=brown!90!black, % borda suave
	fonttitle=\bfseries,
	title=Solução,
	boxrule=0.4pt,
	arc=0pt,
	outer arc=0pt,
	coltitle=black
>>>>>>> 1ff2e7c (Atualizando documentos)
}