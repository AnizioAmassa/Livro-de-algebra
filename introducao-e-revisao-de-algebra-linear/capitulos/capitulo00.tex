\documentclass[../main.tex]{subfiles}
\begin{document}
	
	\chapter*{Prefácio}
	Este documento aborda os conteúdos do curso de Álgebra Linear, ministrado no período 2025.1 do curso Bacharelado em Matemática da Tecnologia e Inovação, do Instituto de Matemática Pura e Aplicada e Tecnologia -- IMPA Tech. O objetivo deste material é auxiliar os discentes do IMPA Tech que estão no primeiro ano da graduação nos estudos da presente matéria, bem como servir de revisão para as demais turmas. Sua construção foi feita com a orientação da professora Nara Bobko, em seu projeto de extensão para elaboração de materiais de estudo para Álgebra Linear.
	
	Assim como o curso de Álgebra Linear do IMPA Tech, este material não apresenta pré-requisitos, a não ser o conhecimento da nossa língua materna e um nivelamento de matemática em conceitos básicos, que podem ser facilmente resgatados durante a leitura.
	
	No capítulo 1, o foco será, principalmente, resgatar e abordar assuntos básicos, necessários ao discente para o aproveitamento completo do curso de Álgebra Linear, e que possivelmente não foram apresentados no Ensino Médio. Isso inclui noções básicas de vetores, bem como de objetos geométricos como retas e planos. Além disso, tem-se um resgate do conceito de matrizes, em uma perspectiva voltada principalmente para matrizes como vetores, e vice-versa.
	
	No capítulo 2 em diante, conceitos da Álgebra Linear são de fato apresentados, iniciando com sistemas lineares, combinação linear e dependência linear. Em seguida, o capítulo 3 apresenta o coração da Álgebra Linear, que são os espaços vetoriais, a partir de ideias como base de um espaço vetorial e ortogonalidade; e o capítulo 4 segue com outro conceito fundamental, que são as transformações lineares. Por fim, no capítulo 5, autovalores e autovetores são apresentados, trazendo aplicações diretas em diagonalização de matrizes, por exemplo.
	
	Espera-se um bom aproveitamento deste material, que apresenta tópicos de Álgebra Linear de forma direta e clara, com algumas motivações geométricas.
	\vfill 
	\begin{flushright}
		\textit{Feito por:}\\
		\textbf{Anizio \& Mariana}
	\end{flushright}
	
	\newpage
	A elaboração deste material foi motivada, majoritariamente, pelas dificuldades pessoais que enfrentei ao cursar Álgebra Linear. Meu objetivo é oferecer um suporte que facilite o acompanhamento do curso, servindo tanto para o aprendizado inicial quanto para revisão. Os exercícios aqui presentes foram selecionados de listas anteriores e de referências clássicas da bibliografia.
	
	É notável que esta disciplina é fundamental. Nas quatro ênfases do IMPA Tech, a Álgebra Linear é totalmente necessária, assim como Cálculo: da Análise no $\mathbb{R}^n$ e em Variedades à Geometria Riemanniana e Teoria Espectral. O domínio desses conceitos é a base para um estudante em qualquer uma das ênfases. 
	
	Para o aprofundamento teórico, recomendo as obras de Elon Lages Lima, Sheldon Axler e Hoffman \& Kunze. São textos densos, mas que recompensam o esforço intelectual do aluno.
	
	Grande parte dos exemplos e observações nesse texto são baseados no espaço real ($\mathbb{R}^n$), mas não se limite a somente esse tipo de conceitos. A Álgebra Linear é muito mais abrangente do que isso. Se tiver interesse em estruturas mais abstratas pesquise sobre espaços duais, de polinômios e de funções. Um exercício valioso durante a leitura é tentar transpor as definições e teoremas apresentados aqui para esses contextos mais abstratos.
	
	\vfill 
	
	\begin{flushright}
		\textit{Feito por:}\\
		\textbf{Anizio}
	\end{flushright}
	
\end{document}

