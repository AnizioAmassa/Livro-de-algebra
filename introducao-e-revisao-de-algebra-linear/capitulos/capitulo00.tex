\documentclass[../main.tex]{subfiles}
\begin{document}
	
	\chapter*{Prefácio}
		Este documento aborda os conteúdos do curso de Álgebra Linear, ministrado no período 2025.1 do curso Bacharelado em Matemática da Tecnologia e Inovação, do Instituto de Matemática Pura e Aplicada e Tecnologia -- IMPA Tech. O objetivo deste material é auxiliar os discentes do IMPA Tech que estão no primeiro ano da graduação nos estudos da presente matéria, bem como servir de revisão para as demais turmas. Sua construção foi feita com a orientação da professora Nara Bobko, em seu projeto de extensão para elaboração de materiais de estudo para Álgebra Linear.
		
		Assim como o curso de Álgebra Linear do IMPA Tech, este material não apresenta pré-requisitos, a não ser um nivelamento de matemática em conceitos básicos do Ensino Médio, que podem ser facilmente resgatados durante a leitura.
		
		No capítulo 1, o foco será, principalmente, resgatar e abordar assuntos básicos, necessários ao discente para o aproveitamento completo do curso de Álgebra Linear, e que possivelmente não foram apresentados no Ensino Médio. Isso inclui noções básicas de vetores, bem como de objetos geométricos como retas e planos. Além disso, tem-se um resgate do conceito de matrizes, em uma perspectiva voltada principalmente para matrizes como vetores, e vice-versa.
		
		No capítulo 2 em diante, conceitos da Álgebra Linear são de fato apresentados, iniciando com sistemas lineares, combinação linear e dependência linear. Em seguida, o capítulo 3 apresenta o coração da Álgebra Linear, que são os espaços vetoriais, a partir de ideias como base de um espaço vetorial e ortogonalidade; e o capítulo 4 segue com outro conceito fundamental, que são as transformações lineares. Por fim, no capítulo 5, autovalores e autovetores são apresentados, trazendo aplicações diretas em diagonalização de matrizes, por exemplo.
		
		Espera-se um bom aproveitamento deste material, que apresenta tópicos de Álgebra Linear de forma direta e clara, com algumas motivações geométricas.
	
\end{document}
