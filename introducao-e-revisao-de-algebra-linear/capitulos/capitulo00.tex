\documentclass[../main.tex]{subfiles}
\begin{document}
	
	\chapter*{Prefácio}
	Este documento aborda os conteúdos do curso de Álgebra Linear, ministrado no período 2025.1 do curso Bacharelado em Matemática da Tecnologia e Inovação, do Instituto de Matemática Pura e Aplicada e Tecnologia -- IMPA Tech. O objetivo deste material é auxiliar os discentes do IMPA Tech que estão no primeiro ano da graduação nos estudos da presente matéria, bem como servir de revisão para as demais turmas. Sua construção foi feita com a orientação da professora Nara Bobko, em seu projeto de extensão para elaboração de materiais de estudo para Álgebra Linear.
	
	Assim como o curso de Álgebra Linear do IMPA Tech, este material não apresenta pré-requisitos, a não ser o conhecimento da nossa língua materna e um nivelamento de matemática em conceitos básicos, que podem ser facilmente resgatados durante a leitura.
	
	No capítulo 1, o foco será, principalmente, resgatar e abordar assuntos básicos, necessários ao discente para o aproveitamento completo do curso de Álgebra Linear, e que possivelmente não foram apresentados no Ensino Médio. Isso inclui noções básicas de vetores, bem como de objetos geométricos como retas e planos.
	
	O segundo capítulo inicia com o conceito de matrizes, que são relacionadas fortemente com sistemas lineares, bem como transformações lineares, coordenadas e diversos outros objetos da Álgebra Linear. Ao final do capítulo, é dada uma noção inicial de combinação e dependência linear.
	
	Em seguida, o capítulo 3 apresenta o coração da Álgebra Linear, que são os espaços vetoriais, a partir de ideias como base de um espaço vetorial e ortogonalidade iremos construir os capítulos posteriores, usando maior parte do conteúdo deste capitulo como linguagem nos seguintes
	
	Quase ao fim, temos o capítulo 4, seguimos com outro conceito fundamental: as transformações lineares. É um dos capítulos mais importantes, se não o mais importante. Compreender totalmente o que está contido nessa parte do texto será vital nas quatro ênfases, pois servirá de base para matérias avançadas no futuro. Destaco uma de cada ênfase: Análise no $\mathbb{R}^n$ (Matemática), Relatividade Geral (Física), Machine Learning (Ciência de Dados) e Computação Gráfica (Ciência da Computação).
	 
	Por fim, o capítulo 5 introduz autovalores e autovetores: um tema novamente fundamental em todas as ênfases. Eles são a chave para a diagonalização de matrizes, processo que permite calcular potências de operadores e resolver sistemas dinâmicos com eficiência. Mais do que apenas ferramentas de cálculo, eles expõem a propriedades interessantes das transformações lineares, mostrando como o espaço se comporta em seus eixos principais.
	
	Espera-se um bom aproveitamento deste material, que apresenta tópicos de Álgebra Linear de forma direta e clara, com algumas motivações geométricas e no espaço real.
	\vfill 
	\begin{flushright}
		\textit{Feito por:}\\
		\textbf{Anizio \& Mariana}
	\end{flushright}
	
	\newpage
	A minha colaboração neste material foi motivada, majoritariamente, pelas dificuldades pessoais que enfrentei ao cursar Álgebra Linear. Maior parte do que abordei aqui foi nos capítulos 3 e 4 (Transformações lineares e suas propriedades). Busquei um viés que facilite o acompanhamento do curso, servindo tanto para o aprendizado inicial quanto para revisão. Os exercícios aqui presentes foram selecionados de listas anteriores e de referências clássicas da bibliografia.
	
	É notável que esta disciplina é fundamental. Nas quatro ênfases do IMPA Tech, a Álgebra Linear é totalmente necessária, assim como o Cálculo. As matérias do primeiro ano são as mais importantes em um contexto geral, o domínio desses conceitos é a base para um estudante em qualquer uma das ênfases. Então, dê o gás.
	
	Para o aprofundamento teórico, recomendo as obras de Sheldon Axler e a do Halmos. Ambos são textos que focam na estrutura de operadores lineares, mais voltados à aplicações diretas em matemática \azul{pura}, como análise funcional. 
	
	Grande parte dos exemplos e observações nesse texto são baseados no espaço real ($\mathbb{R}^n$), mas o leitor não deve se limitar a somente esse tipo de conceitos. A Álgebra Linear é muito mais abrangente do que isso. Se tiver interesse em estruturas mais abstratas pesquise sobre espaços duais, de polinômios e de funções. Um exercício valioso durante a leitura é tentar transpor as definições e teoremas apresentados aqui para esses contextos mais abstratos.
	
	\vfill 
	
	\begin{flushright}
		\textit{Feito por:}\\
		\textbf{Anizio}
	\end{flushright}
	
	\newpage
	A ideia principal desse material é de revisão. Por conta disso, optei por omitir grande parte das demonstrações e incentivar o leitor a tentar provar os teoremas que fossem mais intuitivos ou importantes. Ainda assim, tentei dar alguma motivação, para que o resumo não ficasse tão seco e sem sabor.
	
	Embora esse material não tenha pré-requisitos, é interessante acompanhar com um livro de fato, assim como com as devidas aulas, caso esteja fazendo o curso de Álgebra Linear pela primeira vez. Além disso, recomendo que o aluno que estiver no início dessa ''fase acadêmica'' busque se acostumar com o linguajar matemático. Incentivamos isso com o próprio material, utilizando símbolos como ''$\exists$'', ''$\forall$'', e tantos outros que acompanharão o leitor durante o curso, e talvez nos próximos anos, especialmente se estiver fazendo um curso de Matemática.
	
	Com relação ao material principal de fato, livros rigorosos são muito bons e densos, a exemplo do Hoffman e do Lang. Isso é fortemente recomendado para quem já está acostumado com abstração, ou para quem quer se acostumar para entrar numa matemática mais pura. No entanto, caso o leitor prefira algo mais didático e precise de base, talvez voltado mais à aplicação, recomendo os livros do Reginaldo Santos (GAAL \cite{santos}), do Anton (Álgebra linear com aplicações \cite{anton}) e de Lay/McDonald (Álgebra Linear e Suas Aplicações \cite{lay}), que foram inclusive alguns dos livros que utilizamos para a construção desse material.
	
	Por fim, desejo bons estudos. Apesar de ser muito conteúdo, é uma disciplina básica e essencial tanto para os puristas, quanto para os que buscam aplicação. Aos que estão iniciando, aproveitem essa parte inicial; aos que estão revisando, espero que tirem bom proveito para rever (quase) tudo concisamente.
	
	\vfill 
	
	\begin{flushright}
		\textit{Feito por:}\\
		\textbf{Mariana}
	\end{flushright}
	
\end{document}

