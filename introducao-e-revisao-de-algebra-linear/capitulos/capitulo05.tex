\documentclass[../main.tex]{subfiles}
\begin{document}
	
	\chapter{Autovalores e Autovetores}
		\begin{definicao}
			Seja $T\colon \mathbb{V}\to \mathbb{V}$ um operador linear. Então $x\in\mathbb{R}^n$, tal que $x\neq \bar{0}$, é um \azul{autovetor} de $T$ se
			\[
			\exists \, \lambda\in \mathbb{R} \text{, tal que } T(x)=\lambda x
			\]
			O escalar $\lambda$ é um \azul{autovalor} de $T$. Dizemos que $x$ é um autovetor associado ao autovalor $\lambda$.
		\end{definicao}
		\begin{definicao}
			Seja $A_{n\times n}$ uma matriz. A \azul{equação característica} de $A$ é $\det(\lambda I-A)=0$.
		\end{definicao}
		\begin{teorema}
			Seja $A_{n\times n}$ uma matriz associada a uma transformação linear $T$. Então $\lambda$ é autovalor de $T$ $\Leftrightarrow$ $\det(\lambda I-A)=0$.
		\end{teorema}
		\begin{teorema}
			Seja $A_{n\times n}$ uma matriz triangular. Então os autovalores de $A$ são os elementos da sua diagonal principal (os "$a_{ii}$").
		\end{teorema}
		\begin{teorema}
			Seja $A_{n\times n}$ uma matriz. São equivalentes:
			\begin{enumerate}[label=\roman*)]
				\item $\lambda$ é autovalor de $A$.
				\item O sistema$(\lambda I - A)X=\bar{0}$ tem solução não trivial.
				\item $\exists \, x\neq 0 \text{, tal que } Ax=\lambda x$.
				\item $\exists \, \lambda \in \mathbb{R} \text{, tal que } \det(\lambda I-A)=0$.
			\end{enumerate}
		\end{teorema}
		\begin{teorema}
			Seja $k\in \mathbb{N}^*$.
			
			$x$ autovetor de $A$ associado ao autovalor $\lambda$ $\Rightarrow$ $\lambda^k$ autovetor de $A^k$ associado ao autovalor $\lambda^k$.
		\end{teorema}
	
		\begin{teorema}
			$A_{n\times n}$ invertível $\Leftrightarrow$ $\lambda = 0$ não é autovalor de $A$. 
		\end{teorema}
		
		Recomendo consultar o Teorema 5.1.6 do livro do Anton \cite{anton}, para uma revisão boa de conceitos dados desde o início até aqui, de modo a interligar as afirmações desse último teorema com as demais.
	\section{Diagonalização}
	
\end{document}